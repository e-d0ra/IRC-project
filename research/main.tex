\documentclass[11pt]{article}
\usepackage[a4paper, margin=1.2in]{geometry}
\usepackage{microtype}:
\usepackage{comment}
\usepackage{graphicx}
%\usepackage{mathpazo}
\usepackage{hyperref}
\input{preamble/preamble.tex}

\newcommand{\RNum}[1]{\uppercase\expandafter{\romannumeral #1\relax}}
\title{\textbf{Modelling the Number of Radioactive Particles on a Surface}}
\author{Group 2}

\begin{document}

\maketitle

\section*{Count Rate Integral}

We assume:
\begin{itemize}
  \item $L$ is the activity (or emission rate) of the source.
  \item The source occupies a volume parameterized by $(x_1, y_1, z_1)$ 
        in the bounds $x_1 \in [X_{1,\min}, X_{1,\max}]$, 
        $y_1 \in [Y_{1,\min}, Y_{1,\max}]$, 
        $z_1 \in [Z_{1,\min}, Z_{1,\max}]$.
  \item The detector is centered at $x = x_2$ on the $x$-axis and spans 
        $(y_2,z_2)$ in $y_2 \in [Y_{2,\min}, Y_{2,\max}]$, 
        $z_2 \in [Z_{2,\min}, Z_{2,\max}]$.
  \item The integrand below is an example for a simple $1/r^2$ geometry factor 
        (plus angular dependence).  You may need to adjust powers of $r$ 
        depending on your precise setup (solid angle, flux, etc.).
\end{itemize}

A typical form for the count rate $C$ (number of detections per unit time) can be written as a $5$-dimensional integral:

\[
C \;=\; \frac{L}{4 \pi}
\,\iiint_{\substack{x_1 \in [X_{1,\min}, X_{1,\max}]\\
                   y_1 \in [Y_{1,\min}, Y_{1,\max}]\\
                   z_1 \in [Z_{1,\min}, Z_{1,\max}]}}
\,\iint_{\substack{y_2 \in [Y_{2,\min}, Y_{2,\max}]\\
                   z_2 \in [Z_{2,\min}, Z_{2,\max}]}}
\frac{x_2 + x_1}{\bigl((x_2 + x_1)^2 + (y_2 + y_1)^2 + (z_2 + z_1)^2 \bigr)^{2}}
\,dy_2\,dz_2\,dx_1\,dy_1\,dz_1.
\]

Here:
\[
r^2 \;=\; (x_2 + x_1)^2 \;+\; (y_2 + y_1)^2 \;+\; (z_2 + z_1)^2,
\]
and we have written one possible form of the integrand that places the detector plane at $x = x_2$ (i.e.\ we shift $x_1$ by $x_2$).  

Adjust exponents (e.g.\ $1/r^2$ or $1/r^3$) to match your particular physical model (flux, solid angle subtended, etc.).

\end{document}

