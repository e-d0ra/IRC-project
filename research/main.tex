\documentclass[11pt]{article}
\usepackage[a4paper, margin=1.2in]{geometry}
\usepackage{microtype}:
\usepackage{comment}
\usepackage{graphicx}
%\usepackage{mathpazo}
\usepackage{hyperref}
%% Packages %%

\usepackage[utf8]{inputenc}
\usepackage[T1]{fontenc}
%\usepackage[usenames, dvipsnames]{xcolor}
\usepackage{graphicx}
\usepackage{amsmath, amsfonts, mathtools, amsthm, amssymb}
\usepackage{tikz}
\usepackage{tikz-cd}
%\usepackage{pgfplots}
\usepackage{thmtools}
\usepackage[framemethod=TikZ]{mdframed} % Theorem styling
%\usepackage[explicit]{titlesec}
%\usepackage{fix-cm}
%\usepackage{fancyhdr}
\usepackage[nottoc, numbib]{tocbibind}
\usepackage{bm}

%% Math commands %%

\newcommand{\N}{\ensuremath{\mathbb{N}}}
\newcommand{\R}{\ensuremath{\mathbb{R}}}
\newcommand{\Z}{\ensuremath{\mathbb{Z}}}
\newcommand{\Q}{\ensuremath{\mathbb{Q}}}
\newcommand{\C}{\ensuremath{\mathbb{C}}}
\newcommand{\F}{\ensuremath{\mathbb{F}}}
%\DeclareMathOperator{\ker}{ker}
%\DeclareMathOperator{\im}{im}
%\DeclareMathOperator{\sgn}{sgn}

%% Colours %%

% Rosepine (https://rosepinetheme.com/palette/)
\definecolor{LoveMain}{HTML}{eb6f92}
\definecolor{LoveDawn}{HTML}{b4637a}
\definecolor{TextMain}{HTML}{e0def4}
\definecolor{TextDawn}{HTML}{575279}
\definecolor{GoldMain}{HTML}{f6c177}
\definecolor{GoldDawn}{HTML}{ea9d34}
\definecolor{RoseMain}{HTML}{ebbcba}
\definecolor{RoseDawn}{HTML}{d7827e}
\definecolor{PineMain}{HTML}{31748f}
\definecolor{PineDawn}{HTML}{286983}
\definecolor{FoamMain}{HTML}{9ccfd8}
\definecolor{FoamDawn}{HTML}{56949f}
\definecolor{IrisMain}{HTML}{c4a7e7}
\definecolor{IrisDawn}{HTML}{907aa9}

%% Hyperref %%
\usepackage[]{hyperref}

%% Fancy Header %%

%\renewcommand{\headrulewidth}{0.5}
%\renewcommand{\headrule}{\hbox to\headwidth{%
%\color{GoldDawn!60}\leaders\hrule height \headrulewidth\hfill}}

%% Theorem style %%

\mdfsetup{skipabove=1em,skipbelow=0em, innertopmargin=5pt, innerbottommargin=6pt}

\theoremstyle{definition}

\declaretheoremstyle[
  headfont=\bfseries,
  bodyfont=\normalfont,
  mdframed={ nobreak },
  numbered=no,
]{thmbw}

\declaretheoremstyle[
  headfont=\bfseries,
  bodyfont=\normalfont,
  qed=\diamond,
  numbered=no,
]{thmsub}

\declaretheoremstyle[
headfont=\bfseries,
bodyfont=\normalfont,
numbered=no,
mdframed={
rightline=false,
topline=false,
bottomline=false,
},
qed=\qedsymbol
]{proofbw}


%% Theorems, Lemmas, etc %%
\declaretheorem[style=thmbw, name=Definition, numberwithin=section]{definition}
\declaretheorem[style=thmsub, name=Example, numberwithin=definition]{example}
\declaretheorem[style=thmbw, name=Theorem, sibling=definition]{theorem}
\declaretheorem[style=thmbw, name=Lemma, sibling=definition]{lemma}
\declaretheorem[style=thmbw, name=Corollary, numberwithin=theorem]{corollary}
\declaretheorem[style=thmbw, name=Proposition, sibling=definition]{proposition}

\declaretheorem[style=proofbw, name=Proof]{replacementproof}
\renewenvironment{proof}[1][\proofname]{\begin{replacementproof}}{\end{replacementproof}}

\author{Akira Tanase}


\newcommand{\RNum}[1]{\uppercase\expandafter{\romannumeral #1\relax}}
\title{\textbf{Modelling the Number of Radioactive Particles on a Surface}}
\author{Group 2}

\begin{document}

\maketitle

\section*{Count Rate Integral}

We assume:
\begin{itemize}
  \item $L$ is the activity (or emission rate) of the source.
  \item The source occupies a volume parameterized by $(x_1, y_1, z_1)$ 
        in the bounds $x_1 \in [X_{1,\min}, X_{1,\max}]$, 
        $y_1 \in [Y_{1,\min}, Y_{1,\max}]$, 
        $z_1 \in [Z_{1,\min}, Z_{1,\max}]$.
  \item The detector is centered at $x = x_2$ on the $x$-axis and spans 
        $(y_2,z_2)$ in $y_2 \in [Y_{2,\min}, Y_{2,\max}]$, 
        $z_2 \in [Z_{2,\min}, Z_{2,\max}]$.
  \item The integrand below is an example for a simple $1/r^2$ geometry factor 
        (plus angular dependence).  You may need to adjust powers of $r$ 
        depending on your precise setup (solid angle, flux, etc.).
\end{itemize}

A typical form for the count rate $C$ (number of detections per unit time) can be written as a $5$-dimensional integral:

\[
C \;=\; \frac{L}{4 \pi}
\,\iiint_{\substack{x_1 \in [X_{1,\min}, X_{1,\max}]\\
                   y_1 \in [Y_{1,\min}, Y_{1,\max}]\\
                   z_1 \in [Z_{1,\min}, Z_{1,\max}]}}
\,\iint_{\substack{y_2 \in [Y_{2,\min}, Y_{2,\max}]\\
                   z_2 \in [Z_{2,\min}, Z_{2,\max}]}}
\frac{x_2 + x_1}{\bigl((x_2 + x_1)^2 + (y_2 + y_1)^2 + (z_2 + z_1)^2 \bigr)^{2}}
\,dy_2\,dz_2\,dx_1\,dy_1\,dz_1.
\]

Here:
\[
r^2 \;=\; (x_2 + x_1)^2 \;+\; (y_2 + y_1)^2 \;+\; (z_2 + z_1)^2,
\]
and we have written one possible form of the integrand that places the detector plane at $x = x_2$ (i.e.\ we shift $x_1$ by $x_2$).  

Adjust exponents (e.g.\ $1/r^2$ or $1/r^3$) to match your particular physical model (flux, solid angle subtended, etc.).

\end{document}

